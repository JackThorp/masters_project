% Chapter 

\chapter{Conclusion} % Main chapter title

\label{Conclusion} % For referencing the chapter elsewhere, use \ref{Chapter1} 
We conclude this study by reflecting on our initial objectives:
\begin{itemize}
\item To develop an insight into the 'state of art' of dapp development and the distributed web.
\item To gain understanding of the practical and theoretical limitations of this technology.
\item Using this technology, develop a concrete application to enable cooperative governance at scale. 
\end{itemize}

We feel the first objective has been soundly achieved. We hope it is evident from reading this report that a solid amount of working knowledge has been amassed on the common concepts, processes and tools needed to develop dapps. Or if not, then at least enough to qualify for 'insight'.\\

We also feel that our second objective has been fulfilled. Most of the evidence for this lies in the background knowledge and in the challenges explored in the research and design section. The tools are well understood as are the implications of using them for real applications.\\

The final objective is fulfilled in part. Go-op can only be considered a proof of concept application and, as discussed in our evaluation, certainly a long way from being 'fit for purpose'. Despite this we feel there are many positives to take away from our pursuit of the third objective. Firstly, the amount of new knowledge required to develop a dapp means that the creation of even a simple Ethereum application should be considered a success. Secondly, the process has enabled us to contribute significant work towards establishing the requirements that a scalable co-operative governance application would need to fullfil. Finally, we have been able to identify the technical challenges and potential solutions for future Go-op development that could see it's requirements fulfilled.\\

There is a lot of positive energy and synthesis coming out of the re-decentralised web and co-operative communities at the moment. We hope that this will provide a fertile culture in which Go-op and projects like it can thrive and flower into real usable applications that enable democratic forms of organisation and cooperation to work at scale. Maybe the next era of the web can look different to today. By building trusted democratic enterprises using decentralised technology, applications like Go-op could help Tim Berners Lee's vision of a secure, open and inclusive web come true.\\