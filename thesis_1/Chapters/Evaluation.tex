% Chapter 

\chapter{Evalutation} % Main chapter title
\label{Evaluation} % For referencing the chapter elsewhere, use \ref{Chapter1} 
\section{Technology}
The first objective of this study has been to develop an insight into the current state of distributed web application development using Ethereum and IPFS. This section summarises and expands on the comments made throughout this report about the practical and theoretical limitations of this technology. \\

\subsection{Contract development}
Learning to develop smart contracts in Solidity has been a fun and informative experience, it is quite unlike normal server side development and requires careful attention. As discussed in section \ref{sec:scDesign}, smart contract code needs to be designed so that it can easily adapt over time. This means building highly modular systems defined with robust APIs. At the same time, the gas costs of executing code mean that it is not only important to write efficiently but to also reason further about which logic must necessarily be executed on the blockchain at all.\\

As discussed in the implementation section, the interaction of contracts executing in the EVM is currently limited in places. Most importantly, contracts can't return variable length arrays to one another. The ramification of this issue is that some parts of the public facing API need to be exposed by low level storage contracts. The consequence of spreading the 'backend' API across many smart contracts is that it increases coupling with 'front end' application code. The complexity of 'front end' code increases a little bit for every smart contract in the 'back end' it needs to communicate with. Planned protocol updates mean this problem is likely to disappear eventually. For now, designing smart contract systems is just a little bit trickier.\\

One issue that caused a few headaches during smart contract development was the duplication of contracts across Solidity files. Normally, it's practical to define classes in their own file, which is the approach we tool developing the Go-op smart contracts. However, in order to compile, the definition of any external function called by a contract also needs to be included in the same file. For a contract that actually creates an instance of another contract, the whole contract definition must be included within the file. This is exactly the case with the CoopRegistry and CoopContract (see section \ref{subsec:coopReg}) and was the cause for a few bugs. If a contract is defined in multiple files then keeping those definitions consistent is troublesome. It is easy to change one copy and forget to change another. Solidity provides a fairly crude import statement that doesn't handle import collisions so is tricky to work with. If you are planning to develop smart contracts, watch out for this.\\

Finally a word on testing. Applications are usually built on Ethereum because they require a high degree of security. This make testing them well even more important than usual. Unfortunately the range and maturity Solidity test frameworks is not great. A couple of projects to look out for are Dapple\cite{dapple} and sol-Tester\cite{SolTester}. \\


\subsection{Ethereum clients}
To build a dapp, a developer will need to connect to an Ethereum network using an Ethereum client. Two clients have been explored during Go-op development: geth (a full client written in go) and testrpc (an in memory client 'stub'). Using geth it is possible to connect to any number of separate Ethereum networks including the live public network, the live test network and local private networks. Instructions for setting up a local private network are given in appendix A. Unlike geth, testrpc doesn't actually mine transactions so is much quicker and far less resource intensive, which makes it a good tool for development.\\

The biggest 'blocker' during the project, which took some weeks to resolve properly, stemmed from a 'false positive' error message in geth. After submitting a transaction from the application to the geth client, the client's logs would output the following error message: "transaction (<txdata>) removed from pool: low tx nonce or out of funds." (full posting on stack exchange \cite{GethStackExchange}). Some time was spent trying to understand why there was not enough gas, why the gas-price might be too low, or whether nonces were zero indexed before we discovered that it was in fact a defect in the geth program itself\cite{GethBug}. Despite the error message, transactions were not actually being dropped and did have sufficient funds as well as a correct nonce. It was a false positive error message. Unfortunately a lot of valuable time during the middle of the project was lost by not discovering this sooner. \\

Whilst testrpc is recommended to speed up development it should be used with an understanding that it is not representative of a real production environment. In order to build applications with a good user experience some stage of development and testing should use a production like network. Geth makes it very easy to connect to the testnet and online ether faucets make it very easy to deploy and test smart contract systems in a production parity environment. \\

\subsection{Ethereum Project}
The success or failure of applications like Go-op is of course entirely dependent on the success or failure of the Ethereum platform as a whole. Probably the most important factors to consider when evaluating the likilhood of Ethereum's success are scalability, governance and finance.

\subsubsection{Scalability}
If the Ethereum platform is unable to scale to meet rising demand then dapps running on the network will either become very expensive or very slow. Whilst some applications would be able to endure such conditions but many more would not. \\

How much of a demand increase are we likely to see? This is a difficult question to answer but the short answer is a reasonable amount. It will probably depend a lot on the successes of the early adopters but there are already quite a few of them.\\

Will the platform scale? Currently there are two ways Ethereum are hoping to scale the network. The first is the introduction of a PoS (proof of stake) algorithm called Casper\cite{casper} which would serve as an alternative consensus protocol to proof of work. It is quite a complex idea where miners bet on the inclusion of blocks against the network protocol. PoS is less resource intensive the PoW because it doesn't need to burn CPU cycles to break hash functions. This makes it both better for the environment and more resilient to external factors such as energy prices. More important for scalability is the potential to process transactions faster (conservative block time estimate is 4 seconds, optimistic is sub second \cite{posTPS}). Finally, PoS could make the network more accessible for computers that don't have the high specs required to be a PoW miner which would help to increase network security.\\

A second approach Ethereum may take to scale the network is 'sharding'. A sharded blockchain network is broken down into a number of 'substates' such that no miner actually holds a full copy of the blockchain. Besides having lower storage overheads, network miners also only need to process transactions in their own substate. Because there are multiple substates, more transactions can be processed simultaneously. Of course, the big question is what happens when a transaction occurs between accounts that exist in different substates? This is not an easy problem to solve if the properties of a distributed architecture still need to be maintained but Buterin and the Ethereum team are exploring some novel ideas including the use of hypercubes\cite{ButerinHypercubes}!.

\subsubsection{Governance}
Not too long ago, Mike Hearn published a blog post on Medium explaining his belief that the Bitcoin experiment 'has failed' \cite{HearnBitcoin}. He described Bitcoin as being on 'the brink of technical collapse' and the primary reason being an 'entirely artificial capacity cap of one megabyte per block' which essentially acts as a bottleneck in the system. There are a great number of articles, blog posts and comments debating this point - most of them very emotional and personal. Regardless of whether he is correct or not, it cannot be denied that Mike Hearn's comments have caused greate uncertainty and instability within the Bitcoin system. Not to mention the possible implications of bitcoinXT\cite{bitcoinXT} had it ever taken off. \\

All projects face technical challenges. The difficulty Bitcoin have experienced is not the lack of technical solutions but a failure to debate and agree on them as a community without undermining the entire system, which is surely a question of governance. So, should we be worried about the organisational structure of Ethereum? Well, firstly Ethereum's creator is no mystery, it is Vitalik Buterin and he is well respected and revered across the community. Beyond this, Ethereum has the yellowpaper\cite{wood2014ethereum}, a clearly defined formal specification which means the project is owned less by 'core developers', who understand the main implementation, and more by the community because the spec is far more accessible to understand and develop new clients with. Finally, Ethereum is a crowd funded project so should be a lot more accountable to its funders. This is a different dynamic than that of Bitcoin.\\

\subsubsection{Finance}
Neither the Ethereum Foundation or EthDev turn a direct profit from developing Ethereum because essentially, the network is a public good. This means the project faces the difficult question of 'who is going to pay for it?s maintenance?'. The point is well made by Vinay Gupta in his talk at DevCon1\cite{GuptaFinance}. Despite Vinay's optimism given the vast amount of start up capital raised by the crowd fund (somewhere in the region of \$18 million), he doesn't give concrete answers as to a long term revenue model. It is surely encouraging to see members of the Ethereum team thinking about these problems fairly early on. The huge amount of funds Ethereum based projects, such as TheDAO, are generating and the widespread interest of the financial industry suggest that there is probably enough invested capital to secure the platforms continued development for some time to come.

\subsection{Frameworks}
A number of frameworks, including Meteor, have become popular within the community for dapp development. It is our belief, after trialing a couple of them and using Meteor extensively, that there are no compelling reasons for adopting a new framework specifically for dapp development. From our experience, there are three special requirements for dapp development:

\begin{enumerate}
\item A compiler plugin similar to deployScript.js (see section \ref{subsec:deployScript}) that is able to generate the web3.js 'handles' needed for contract interaction.
\item Some kind of account management tool or 'wallet'.
\item A framework for testing, developing and deploying Solidity smart contracts. 
\end{enumerate}

The first requirement is the main offering for most dapp frameworks. In our opinion, developing your own smart contract compiler plugin is simple if you are familiar with Ethereum and a very instructive process if you are not. If you already have a favourite web development framework, build tool etc you are comfortable with, integrating the first requirement should not be too demanding.\\

There are a few tools for in browser wallets such as eth-lightwallet\cite{ethLightwallet}. Most are general libraries that can be used regardless of framework choice. The other option for account management is Mist which just exposes an API to the application and doesn't require any kind of dedicated framework..\\

Finally, writing and testing solidity contract code. This is very important but does not need to form part of a framework for 'front end' development. Frameworks specifically for Solidity development are likely to be very useful. There are a number of possible solutions including the Mix IDE and Dapple here.\\

At the end of the day, developing professional SPAs (which dapps must necessarily be) is a serious engineering process regardless of integration with Web3 clients etc. The biggest factor for choosing a framework or tool be it's suitability for SPA development. Meteor is popular because many developers think this is such a tool regardless of specific support for dapps development.\\

\subsection{IPFS}
IPFS was very straightforward to use and was not the cause of any problems. Hopefully, planned integration with browsers and the Swarm project will make it even easier to use in future both within dapps and in general. Watch this space.\\

\section{Solution}
The second objective of this study has been to develop a governance application for large co-operative enterprises. This section evaluates the Go-op solution and whether it is 'fit for purpose'.\\

\subsection{Cost analysis}
All Ethereum transactions cost Ether. Understanding how much running Go-op costs is important when assessing its suitability for co-operatives. The cost of 'deploying' the static contracts such as registries etc is only incurred once and is therefore constant and negligable. The operations that require investigation are those carried out by users every time they interact with the system. In Go-op, transactions are fired for the following processes:
\begin{itemize}
\item Registering as a new user
\item Changing user information
\item Creating a new co-operative
\item Joining a co-operative i.e. membership
\item Creating a proposal (includes cost of using the Ethereum Alarm Clock service)
\item Voting on a proposal
\end{itemize}

Calculating the cost of a transaction in GBP requires knowledge of three variables: units of gas consumed, price of gas and Ether/GBP conversion rate. The cumulative gas consumption for each operation has been measured with an instance of Go-op deployed on the Morden testnet (accessing such measurements is easily done using a blockchain explorer such as Ethercamp\cite{EtherCamp}). Gas prices and exchange rates are constantly fluctuating but at the time of writing gas-price is set at around 23 GWei  (or Shannon) per unit and the price of 1 Ether is just below \pounds 10. Using this information we can calculate the cost of each of the above processes:\\
\begin{table}
\begin{center}
\begin{tabular}{ |c|c|c|c|c| } 
 \hline
 operation & gas price (Eth) & gas consumption & cost (Eth) & cost (GBP) \\ 
 \hline
 Coop creation & 0.000000023228 & 739553 & 0.01717833708 & 0.17 \\
 \hline
 Membership & 0.000000023228 & 162791 & 0.00378130934 & 0.04 \\ 
  \hline
 Creating a proposal & 0.000000023228 & 107698 & 0.00250160914 & 0.03 \\ 
  \hline
 Voting & 0.000000023228 & 47308 & 0.00109887022 & 0.01 \\ 
 \hline
\end{tabular}
\caption {Table of costs for running key Go-op operations}
\label{table:costs}
\end{center}
\end{table}

As table \ref{table:costs} shows, the biggest cost in Go-op is currently creating a new cooperative, which costs about seventeen pence (not a lot). Joining as a member costs the user four pence whilst creating a proposal and voting on it costs only three pence and one pence respectively. Although these are trivial amounts, a co-operative using Go-op can of course subsidise the individual costs of governance participation by issuing credit (similar to printer credit). As it stands, the order of the costs involved with using Go-op are totally manageable. Are prices likely to rise significantly in the future? This is difficult to answer but if anything, the introduction of proof of stake consensus and sharding (if it happens) are likely to reduce prices.\\

\subsection{Security analysis}
To be fit for use Go-op needs to be secure. In our opinion the most pressing security concerns with the current state of development are the protection of user information or privacy, contract code verification, and member identification.\\

\subsubsection{Member Identification}
Currently Go-op allows any registered user to join a co-operative so long as they pay their fee and agree to the terms and conditions. Unfortunately, this means it is trivial for any co-operative member to create multiple Ethereum accounts, register multiple user aliases and join a cooperative multiple times to achieve disproportional representation in voting. The only formal barrier to prevent this kind of attack is the cost of repaying the membership fee which is unlikely to prevent a motivated 'attacker'. The wider problem of identity management and some potential solutions are discussed at greater length in section \ref{sec:identity}. The most workable solution for Go-op would probably be the 'curator' approach taken TheDAO. Go-op would enable users to elect a coop member (whose account we assume is already verified) whose role would be to validate the unique identity users applying for membership. The validation of identity would happen as part of some 'off chain' process. Joining a cooperative would still work in the same way but would only be confirmed following a signed transaction from the 'curator'. The cooperative could even arrange to pay a fee to the 'curator' every time they validate a new member.\\

\subsubsection{Code Verification}
Go-op currently lacks a significant amount of code verification. There are no tests for the smart contract system which means it is hard to guarantee its correct behaviour. The absence of smart contract tests so far is a result of time constraints and access to convenient testing frameworks. Any serious proposal for a democracy application ought to be tested rigorously and possibly even formally verified. As part of a testing process we would expect the smart contract system to be re-architected somewhat to improve the 'separation of concerns' and make the interactions between contracts simpler. The current system of a many to many relationship captured across three contracts each with public facing APIs and no common controller is very susceptible to creating data inconsistencies. The Five Types model was not fully realised in this first PoC because of the inability to pass key data types between contracts. When this limitation is addressed in future updates to the EVM a proper implementation of this design pattern should be attempted.

\subsubsection{Privacy}
Whilst Go-op currently only holds a limited amount of user data i.e. a name and email address, it doesn't store that data in encrypted format. This means that user details as well as the fact of membership are publicly accessible. Further to this, member voting behaviour is also publicly visible. From our discussions with co-operative businesses it is apparent that higher levels of privacy and protection are required than currently supported. To address this, any future development should explore the approaches introduced in section \ref{sec:blockchainprivacy}.

\subsection{Performance Analysis}
The usability of Go-op can start to be improved in a few ways. Firstly, through the introduction of more controller and 'application logic' smart contracts, the number of individual transactions, and therefore time to execute a single user process, can be reduced. Secondly, an optimistic UI should be explored that responds instantly to user interaction and allows the fluid use of the application without waiting for transactions to be processed. Thirdly, caching could be explored to the 'front end' database to improve query speeds. The effectiveness of caching in this architecture may not be significant however, because the server i.e. the Ethereum client runs locally, so there is no network that direct blockchain read requests need to travel across. Finally, the Ethereum Alarm clock service may need to be replaced for a lazy evaluation approach to make it easier for members to specify proposal closing in 'real date' terms.\\

\subsection{Fit for purpose}
The security and performance issues with Go-op are important but not insurmountable and have implementable solutions. The key question in evaluating the 'product' is whether it actually meets real user requirements and what evidence there is to support such a claim. From our research on cooperative governance it is apparent that there are three broads requirements an application such as Go-op needs to fulfil - membership management, proposal development and voting. \\

\subsubsection{Membership Management}
Membership management is more than simply the storage of basic member data, it includes relationship development, communication and attribute collection. This requirement became clear to us when talking with John Atherton, head of membership at Co-operatives UK. He explained that the main purpose of their existing membership database was customer relationship management and research. It is important for Coops UK to attach additional data points to their member profiles such as tracking which Coops UK services they have used or which events they have attended. Go-op does not fulfil the membership management requirement. Although we don't need to build an entire CRM (Customer Relationship Management) system into the blockchain to perform distributed governance, to be a serious option, plugins that bridge between Go-op and popular CRM solutions, such as CiviCRM\cite{CiviCRM} or SalesForce\cite{SalesForce}, will need to be developed.\\

\subsubsection{Proposal Development}
To fully realise the vision of large scale distributed governance, digital proposal development tools need to be developed that allow members to collaboratively create, discuss, version, amend and finalise proposals. The role of Go-op in this process should be to create secure and permanent audit trail of the process in the Ethereum blockchain. Developing such a component for Go-op will be a big development task unto itself and will still require a registry holding the digital identities of valid participants.\\

\subsubsection{Voting}
Out of the three governance requirements identified, voting is currently the most supported by Go-op. We have developed a working PoC that demonstrates some type of voting process is possible to host on Ethereum. The main concerns here are again the lack of a verified register of valid participants and the privacy of votes. Despite this, serious proposals to address these issues have been presented and we feel that the challenges are not insurmountable.\\


